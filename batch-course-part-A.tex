%!TEX root = /Users/kevindunn/ConnectMV/Projects/2010/10-010-GSK-batch-course/slides/batch-course.tex

%-------------------------------------------------
% Sources and resources
%-------------------------------------------------
\begin{frame}\frametitle{Sources and resources}

These notes are based on ConnectMV's experience with batch systems, but also the extensive research into batch data analysis done at McMaster University

\begin{enumerate}

	\item	Nomikos, ``Statistical Process Control of Batch Processes'', Ph.D thesis, McMaster University, 1995, {\scriptsize http://digitalcommons.mcmaster.ca/opendissertations/1667/}
	
	\item	Garcia-Mu\~noz, ``Batch process improvement using latent latent variable methods'', Ph.D thesis, McMaster University, 2004 {\scriptsize http://digitalcommons.mcmaster.ca/opendissertations/1596/}
	
	\item	Wold, Kettaneh-Wold, MacGregor, Dunn, ``Batch Process Modeling and MSPC'', \emph{Comprehensive Chemometrics}, 2009, {\scriptsize http://dx.doi.org/10.1016/B978-044452701-1.00108-3}
\end{enumerate}
\end{frame}

%-------------------------------------------------
\section{What do we learn from our data?}
%-------------------------------------------------

\begin{frame}\frametitle{What do we learn from our data?}

\begin{enumerate}
	\item {\bf \color{myGreen}Improve/confirm} process understanding
	
		 How do we learn from our data?
\begin{itemize}

	\item	see which variables behave similarly (clustering) \pause
	\item	confirm which phenomena have greatest effect in the data 
	\begin{itemize}
		\item	high variability appears in first LVs
		\item	lower variability phenomena later
	\end{itemize}\pause

	\item	which variables have most strong influence on variability in that component
	\begin{itemize}
		\item variables with high loadings
	\end{itemize}\pause

	\item	interpret latent variables
	\begin{itemize}
		\item	sometimes we can
		\item	helps us when we think in latent variable space
	\end{itemize}
\end{itemize}
\end{enumerate}
\end{frame}

\begin{frame}\frametitle{What do we learn from our data?}

\begin{enumerate}
	\setcounter{enumi}{1}
	\item {\bf \color{myGreen}Troubleshooting}
	\begin{itemize}
		\item 	confirm a known problem occurred (SPE, \( T^2 \), \( t_a \))
		\item 	use contribution plots to diagnose
		\begin{itemize}
			\item 	what went wrong?
			\item 	when did it go wrong (batch)? 
		\end{itemize}\pause
		
		\item 	use interpretation of \( t_1, t_2 \) to explain high/low score values\pause
		\item 	use engineering judgement to fix problems from insight gained
	\end{itemize}
\end{enumerate}
\end{frame}
	
\begin{frame}\frametitle{What do we learn from our data?}

\begin{enumerate}
	\setcounter{enumi}{2}
	\item {\bf \color{myGreen}Improve/optimize} a process
	\begin{itemize}
		\item 	how can we move in the latent variable space?
		\item 	what are the causal variables to adjust to move in \( t_1, t_2, \ldots \)
		\item 	how do we do latent variable DOE's to fill in gaps
	\end{itemize}	
	These all rely on the concept of \alert{``model inversion''}
\end{enumerate}
\end{frame}

\begin{frame}\frametitle{What do we learn from our data?}

\begin{enumerate}
	\setcounter{enumi}{3}
	\item {\bf \color{myGreen}Predictive modelling} 
	\begin{itemize}
		\item 	rely on PLS models
		\item 	e.g. inferential sensors provide real-time prediction of \( \mathbf{Y} \), our final quality attributes
		\begin{itemize}
			\item 	PLS handles colinearity, missing values, noisy \( \mathbf{X} \)-variables
			\item 	these 3 issues cannot be dealt with in ordinary least squares
		\end{itemize}
	\end{itemize}	
\end{enumerate}
\end{frame}

\begin{frame}\frametitle{What do we learn from our data?}

\begin{enumerate}
	\setcounter{enumi}{4}
	\item {\bf \color{myGreen}Process monitoring} 
	\begin{itemize}
		\item 	build a model from ``in control'' operation: make sure we remain there
		\item 	don't need to take extra action if we remain in that space
		\item 	do adjust process if trending out of space; we have
		\begin{itemize}
			\item 	SPE, \( T^2 \) and \( t_a \) score limits to monitor against
		\end{itemize}
		
		\item 	use contribution plots to diagnose problems
	\end{itemize}	
\end{enumerate}
\end{frame}

\begin{frame}\frametitle{What do we learn from our data?}

\begin{enumerate}
	\setcounter{enumi}{4}
	\item 	{\bf \color{myGreen}Process monitoring} (cont'd)	

	\vspace{10pt}	
			Advantages over \emph{univariate} monitoring:
		
			\begin{itemize}
				\item 	we monitor \textbf{many} raw variables with summarized scores, SPE, and \( T^2 \)
				\item 	i.e. we actually have \emph{fewer} control charts with multivariate monitoring
				\item 	multivariate charts interpreted in the same way as univariate charts
				\item 	i.e. no extra operator training required
				\item 	ordinary Shewhart charts do not give any help with diagnosis
			\end{itemize}	
\end{enumerate}
\end{frame}

\begin{frame}\frametitle{Tools to learn from our data}

\begin{enumerate}
	\item 	\alert{Generic}: applicable to all types of processes
	\item 	\alert{Informative}: easy to use by untrained staff for decision making
		\begin{itemize}
			\item clear and quick action possible
		\end{itemize}
	\item 	\alert{Simple to implement}: straightforward calculations and fast
\end{enumerate}

\vspace{1cm}

\begin{exampleblock}{}<2->
\centering{\large{\color{myOrange} Latent variable methods meet these criteria}}
\end{exampleblock}
\end{frame}

%-------------------------------------------------
\section{General multivariate review}
%-------------------------------------------------

\begin{frame}\frametitle{Justification for latent variable methods}

\begin{enumerate}
	\item	High dimensionality in data we collect, e.g. NIR spectra
			\begin{center}
				\includegraphics[width=6cm]{images/high-dimensionality-data.jpg}
			\end{center}
			
			
	\item	High correlation between the variables (duplicate info)
			\vspace{6pt}
			\begin{columns}
				\column{.30\textwidth}
				\begin{center}
					\includegraphics[width=3cm]{images/high-correlation-between-variables.png}
				\end{center}
				
				\column{.7\textwidth}
				Not a bad thing, but ordinary least squares cannot handle high correlations
				\begin{itemize}
					\item	we resort to variable selection					
					\item	we risk omitting important variables
					\item	using all data: get stronger signal from many noisy variables
				\end{itemize}
				
			\end{columns}
			
\end{enumerate}
\end{frame}

\begin{frame}\frametitle{Justification for latent variable methods}

\begin{enumerate}
	\setcounter{enumi}{2}
	\item	Missing data
	
			\begin{center}
				\includegraphics[width=6cm]{images/missing-data.png}
			\end{center}
			
			Batch data sets: usually not an issue for complete, historical batches			
			
	
	\item	Large number of samples
	
			\begin{itemize}
				\item	modern computer hardware
				
				\item	smart algorithms (build models from smaller data groups)
			\end{itemize} 
\end{enumerate}
\end{frame}

\begin{frame}\frametitle{Quick review of multivariate concepts: PCA}

	\begin{block}{Mathematical objective}
		PCA: find me the best summary of my data, \( \mathbf{X} \), with the fewest number of summary variables, called scores, \( \mathbf{T} \).
	\end{block}
	
	\vspace{18pt}

	\begin{center}
		\includegraphics[width=8cm]{images/reduce-data-X-to-scores-T.png}
	\end{center}
	
\end{frame}

\begin{frame}\frametitle{What does PCA do?}

	It finds directions that best explain the data.  Also called
	
	\begin{itemize}
		\item  	``directions of greatest variance''
		\item	``loadings and scores''
		\item	``components''
		\item	``latent variables'' (LVs)
	\end{itemize}

	\begin{center}
		\includegraphics[width=\textwidth]{images/geometric-PCA-3-and-4-centered-with-first-component.png}
		% images/geometric-interpretation-of-PCA.svg
	\end{center}
	
	PCA finds the LVs so that LV1 explains more than LV2, which explains more than LV3, \emph{etc}.
\end{frame}

\begin{frame}\frametitle{What is a latent variable?}

	\textbf{\emph{Example}}: your health
	
	\begin{itemize}
		\item	There isn't a single measurement of ``health''.  It's an abstract concept: 
		
				\begin{itemize}
					\item	blood pressure, cholesterol, blood sugar, temperature, heart rate \emph{etc}
				\end{itemize}				
				
		\item	combine these values in some way (by a trained professional) to come up with some level of ``health''
	\end{itemize}
	
	\pause
	
	{\color{myGreen}{We can do this with any system!}}
\end{frame}

\begin{frame}\frametitle{What is a latent variable?}

	\textbf{\emph{Example}}: room temperature measured at 4 points
	
	\begin{center}
		\includegraphics[width=6cm]{images/room-temperature-plots.jpg}
		% images/room-temperature-plots.py ---> PNG ---> cropped ---> saved as JPG
	\end{center}
	
	\begin{itemize}
		\item	a single phenomenon occurring, so \textbf{not 4 independent}  measurements
		
		\item	can be reduced down to 1 latent variable
		
				\begin{itemize}
					\item	just use the average
					
					\item	that's exactly what PCA does in this example
				\end{itemize}
		
	\end{itemize}
	
	
	% FUTURE: come back to pencil notes in Red McMaster Sketchbook binder and add the part that shows the weighted sum calculation for the scores

\end{frame}

%-------------------------------------------------
\section{Batch systems}
%-------------------------------------------------
% Related concepts: phases, Z, trajectories, alignment, recipes, operators, manual steps, unfolding, charge reactor

\begin{frame}\frametitle{Batch systems: concepts and unique features}

\begin{itemize}
	\item 	Start point, an end point, and time-based evolution of tags (variables) in between 
	
	\item 	Operate essentially in open-loop i.t.o. final quality attributes (FQAs).  Basic, low-level feedback. \pause
	
	\item 	Quality measurements made afterwards, usually in a lab
	
			\begin{itemize}
				\item	used to decide batch disposition
				\item	often used to adjust next batch (batch-to-batch control)
			\end{itemize}\pause
			
	\item	Sequenced very accurately
	
	\item	Phase-based sequencing

\end{itemize}
\end{frame}

\begin{frame}\frametitle{Batch systems: concepts and unique features}

\begin{itemize}
	
	\item	Nonlinear relationship between variables; relationship changes with time \pause
	
	\item	Poor mechanistic models available (understandable)
	
			\begin{itemize}
				\item	continuous: one or two things happening all the time
				\item	batch: series of phases/events and relationships differ between (and within!) phases
			\end{itemize} \pause
			
	\item 	Past history within a batch affects the future
	
			\begin{itemize}
				\item	initial conditions: affect trajectories and FQAs								
				\item	trajectory changes from operator affect FQAs
				\item	actions taken can affect some period or all of batch, or,
				\item	even have no effect - relevant when monitoring a batch
			\end{itemize}
\end{itemize}
\end{frame}

\begin{frame}\frametitle{Batch systems: terminology for these notes}

\begin{description} 
	
	\item[ \( N \): number of batches] 
	
		\begin{itemize}
			\item	literature uses \( I \)
		\end{itemize}
		
	\item[\( K \): number of tags] 
	
		\begin{itemize}
			\item	 literature uses \( J \)
		\end{itemize}
	
	\item[\( J \): number of time steps, ] 
	
		\begin{itemize}
			\item	 literature uses \( K \)
		\end{itemize}
\end{description}

We aim for consistency with general latent variable methods: \( N \times K \times J \)

\begin{center}
	\includegraphics[width=7.2cm]{images/batch-data-cube.png}
\end{center}

\end{frame}

\begin{frame}\frametitle{Batch systems: data representation}

When retrieving batch data from computerized systems:
\begin{enumerate}
	\item	One batch per sheet in a spreadsheet, with batch ID
			
			\includegraphics[scale=0.75]{images/batches-in-spreadsheets.png}
	
	\item	One batch per CSV file
\end{enumerate}
\end{frame}

\begin{frame}\frametitle{Batch systems: data representation}

\begin{enumerate}
	\setcounter{enumi}{2}
	\item	Stacked batches of equal duration in a single file

			\begin{center}
				\includegraphics[width=0.9\textwidth]{images/batch-data-layers-into-page-and-unfolded-aligned}
			\end{center}
					
\end{enumerate}
The extra batch ID column is optional.  Some batch software require the data input this way.
\end{frame}

\begin{frame}\frametitle{Batch systems: data representation}

\begin{enumerate}
	\setcounter{enumi}{3}
	\item	Stacked, \textbf{but unaligned} batches, in a single file (common)
			
			\begin{columns}
				\column{.3\textwidth}
					\begin{itemize}
						\item	must also include a batch ID column

						\item	phased recipes: append a ``phase ID'' column within each batch (not shown)
					\end{itemize}
					
				\column{0.7\textwidth}
				
					\begin{center}
						\includegraphics[width=\textwidth]{images/batch-data-layers-into-page-and-unfolded.png}
					\end{center}
					
			\end{columns}
\end{enumerate}
MATLAB can handle all the formats shown above.
\end{frame}

\begin{frame}\frametitle{Batch systems: data representation}
	
	\begin{itemize}
		\item	It is very common that the time-dimension, \( J \), is unequal	
		
		\item	We deal later with alignment: equalizing the time-dimension for all batches
	\end{itemize}
\end{frame}

\begin{frame}\frametitle{Batch systems: visualize the trajectory data}
	
	\begin{columns}
		\column{.40\textwidth}
			{\color{myOrange}{\textbf{Unaligned data}}}
 			\vfill
			\includegraphics[height=0.8\textheight]{images/unaligned-trajectories-many-batches.png}
			% From Cecilia Rodrigues thesis, used with permission (see gmail email in February 2011)
		
		\column{.60\textwidth}
			{\color{myOrange}{\textbf{Aligned data}}}
			\vfill
			\includegraphics[height=0.8\textheight]{images/aligned-trajectories-many-batches-yeast.png}
			% Using rev 85:95b49ee15280 			
			% data = load('/Users/kevindunn/ConnectMV/Courses/Batch/datasets/yeast/yeast-data.mat');
			% b.tagNames = {'Ethanol', 'Temperature', 'Molasses', 'NH3 in feed', ...
			%               'Air flow', 'Tank level', 'pH'};
			% b.batchnames = {'rB','rC','rI','rM','rN','rQ','rR','rT','rV','rX','rZ', ...
			%                'ra','rb_1','rc_1','rd','re','rf','rg','rh','ri_1','nA', ...
			%                'tD','nG','nH','tJ','nL','nO','tP','nU','nY','rj','rk','rl'};
			% b.nBatches = 33;
			% batch_X = block(data.X, 'X', 'batch', 'tagNames', b.tagNames, 'nBatches', b.nBatches);
			% plot(batch_X, 'raw', 2, 3) % but used a limited set
	\end{columns}
	\todo{Add software instruction}	
\end{frame}

%-------------------------------------------------
\section{Batch analysis using feature extraction}
%-------------------------------------------------

\begin{frame}\frametitle{Feature extraction methods: overview}

\begin{itemize}
	\item	Unaligned data can be a challenge (more later).
	
	\item	One alternative to alignment, if we want to get started right away: 
	
			\begin{itemize}
				\item	{\color{myOrange}{\emph{extract features}}} from each phase in each batch
				
				\item	assemble features within a row
				
				\item	build ordinary PCA on \( \mathbf{X_f} \) or PLS: \( \left\{ \mathbf{Z} \,\,\text{and}\,\, \mathbf{X_f}\right\} \stackrel{\text{PLS}}{\longmapsto} \mathbf{Y} \)
			\end{itemize}
			
			\todo{8. Features}
\end{itemize}
\end{frame}

\begin{frame}\frametitle{Which features to extract}

\begin{itemize}
	\item	Extract {\color{myGreen}{\emph{within each phase}}} or within {\color{myGreen}{\emph{specified windows}}}. 
	
	\item	Extract for one, some, or \textbf{all} tags:
\end{itemize}\pause

\begin{itemize}
	\item	average value \hfill {\color{myOrange} $\leftarrow$ easiest, most useful feature}
	\item 	median value
	\item	integrated area under a tag (often makes engineering sense)
	\item	standard deviation, (useful if tag should be constant)
	\item	slope of curve\pause
	\item	energy or mass balance calculation over a phase
	 		\begin{itemize}
	 			\item	e.g. heat released by a reaction should be taken up by the cooling water
	 			\item	an imbalance can point out a problematic batch
	 		\end{itemize} \pause
	\item	value of a trajectory at start/end of phase
	\item	total time for a phase to complete (goes into \( \mathbf{Z} \))
\end{itemize}
\end{frame}

\begin{frame}\frametitle{How to use feature-based models}

\begin{itemize}
	\item	Many features extracted: often \( \sim \) 80 to 100 columns; many are not useful
	
	\item	Used in an ordinary PCA or PLS.  All the usual tools apply:
	
			\begin{itemize}
				\item	VIP: finds most import variables for explaining FQAs \( (\mathbf{Y}) \)
				
				\item	loadings and weights: used to interpret any clusters in the scores
				
				\item	\( R^2 \) per variable, small weights: eliminate less useful features
			\end{itemize}\pause
	
	\item	Learn about the batch: {\small e.g.: variability in cooling water temperature related to poor FQA}
	
	\item	Troubleshooting an unusual batch 

	\item	By extension, use troubleshooting information to improve future batches \pause
	
	\item	Can be used for monitoring: we'll come back to this
\end{itemize}
\end{frame}

\begin{frame}\frametitle{Disadvantages of feature-based models}

\begin{itemize}
	\item	Smoothly varying trajectories within a phase
	
			\begin{itemize}
				\item	no distinct features and hard to quantify
			\end{itemize}
			
		\todo{9. Nylon trajectories}
	
	\item	Depends on engineer's prior knowledge
	
	\item	Subtle defects and broken correlations between variables 
	
			\begin{itemize}
				\item	hard/impossible to detect and quantify
			\end{itemize}
			
			\todo{10. plot of subtle defect }

	\item	Feature value may not exist in an abnormal batch: so it's missed
	
\end{itemize}
\end{frame}

\begin{frame}\frametitle{Class example: using feature-based models}

	\begin{itemize}
		
		\item	Feature example for this batch process works well
		
				\begin{itemize}
				
					\item	4 distinct phases with ``sharp'' trajectories
				
				\end{itemize}

	\end{itemize}

	\begin{center}
		\includegraphics[width=\textwidth]{images/fmc/fmc-raw-trajectories.png}
	\end{center}

\end{frame}

\begin{frame}\frametitle{Class example: using feature-based models}

	\begin{itemize}
	
		\item	Features already extracted for you in the data file
	
	\end{itemize}

	\begin{center}
		\includegraphics[width=\textwidth]{images/fmc/fmc-phases-4-trajectories.png}
	\end{center}
	
\end{frame}

\begin{frame}\frametitle{Class example: using feature-based models}

	\begin{itemize}
	
		\item	Features already extracted for you in the data file
	
	\end{itemize}

	\begin{center}
		\includegraphics[width=\textwidth]{images/fmc/fmc-features-extracted.png}
	\end{center}

\end{frame}

\begin{frame}\frametitle{Class example: model structure for features}

	\begin{itemize}
		\item	Features compiled into a multiblock PLS model
		
				\begin{itemize}
					\item	\( \mathbf{Z} \): recipe information and composition of material charged (11 variables)
					
					\item	\( \mathbf{X} \): the 13 extracted features
					
					\item	\( \mathbf{Y} \): 12 final quality variables
				\end{itemize}
	\end{itemize}
	
	\begin{center}
		\includegraphics[width=\textwidth]{images/fmc/fmc-features-multiblock.png}
	\end{center}
\end{frame}

\begin{frame}\frametitle{Class example: model results}
	
	\begin{itemize}
		\item	Score plot shows relatively clear separation between off-spec batches (red circles)
	\end{itemize}

	\begin{center}
		\includegraphics[width=\textwidth]{images/fmc/fmc-score-plot-features.png}
	\end{center}
\end{frame}

\begin{frame}\frametitle{Class example: model results}
	
	\begin{itemize}
		\item	Loading plot for \( \mathbf{Z} \) and \( \mathbf{X} \) blocks
	\end{itemize}
	
	\begin{center}
		\includegraphics[width=\textwidth]{images/fmc/fmc-features-loading-plot.png}
	\end{center}
	
\end{frame}

\begin{frame}\frametitle{Class example: model results}
	
	\begin{itemize}
		\item	VIP for \( \mathbf{Z} \) and \( \mathbf{X} \) blocks. Agrees with loadings.
	\end{itemize}
	
	\begin{center}
		\includegraphics[width=\textwidth]{images/fmc/fmc-VIP-values-features.png}
	\end{center}
\end{frame}

\begin{frame}\frametitle{Class example: what was learned?}
	
	To get good quality batches: operate at high \( t_1 \) and high \( t_2 \).  From loadings plot:
	
	\begin{itemize}
		\item	he plot shows that high rates of increase (slopes) of power, torque, and temperature in phase 1, as well as low solvent level in the collector tank in phase 3 (equivalent to low initial solvent in the wet cake) all relate to a good batch outcome
		% High t1: lower JTempSP slope:  But, JtempSlope is between -0.15 and -0.05, so a lower JTempSlope implies 
		% operation toward -0.15, ie temperature decreases more rapidly for the good batches. This might appear conflicting with the w*1 line plot, which shows that higher values of temperature SP lead to better batches.  How can a batch with a steeper decreasing slope have high value of temperature SP?  Well, the key insight is that temperature SP is always going to END at the same temperature value at the end of phase 2.  Having a steeper, more negative slope, implies that the batch must start at a higher temperature earlier in the batch.  This agrees exactly with the w*1 line plot.
	\end{itemize}
	
	Can be confirmed by looking at what trajectories for some good quality batches.
	
	\todo{Plot good quality batches}
	
\end{frame}


%!TEX root = batch-course.tex

\begin{frame}\frametitle{Batch PCA example: SBR}
	
	\begin{itemize}
		\item	Simulated data from \emph{large} mechanistic model for styrene butadiene rubber
		
		\item	Good to start learning from this case study
		
		\item	Simulated to contain ``normal operating conditions''
		
				\begin{itemize}
					\item	2 main problematic batches
					
					\item	same fault, but starting at different times
				\end{itemize}
	\end{itemize}
	
\end{frame}

\defverbatim[colored]\loaddata{%
\begin{lstlisting}[basicstyle=\tt\scriptsize,frame=none,language=C++,linewidth=.9\textwidth]

data = load('datasets/SBR.mat');   % X is in a  10600 x 9 matrix
Y_data = data.Y;
   
% Specify the data dimensions
nBatches = 53;
nTags = 6;
nSamples = 200;  % not required
tagNames = {'Reactor temperature', 'Cooling water temp', ...
            'Reactor jacket temperature', 'Latex density', ...
            'Conversion', 'Energy released'};
        
% Create a batch block first: tell it how many batches 
% there are in the aligned data
% Ignore tags 1, 2, 3 (noisy and uninformative)

batchX = block(data.X(:, 4:9), 'X', 'batch', ...
               'tagNames', tagNames, 'nBatches', nBatches);
\end{lstlisting}}

\begin{frame}[fragile,containsverbatim]\frametitle{SBR: load the data}
	\loaddata
\end{frame}

\begin{frame}\frametitle{SBR: raw data}
	
	\begin{itemize}
		\item	Batches: \( N = 53 \); Tags: \( K = 6 \); Time steps: \( J = 200 \)
		
	\end{itemize}
	
	\begin{center}
		\includegraphics[width=\textwidth]{images/sbr/SBR-raw-data-trajectories.png}
	\end{center}
	{\color{myOrange}{\texttt{plot(batchX, 'raw', 2, 3)}}}
\end{frame}

\begin{frame}\frametitle{SBR: build model}
	
	\begin{itemize}
		\item	Start with 2 to 3 components: \emph{just to see what's going on}
		
		\item	{\color{myOrange}{\texttt{batch\_PCA = lvm(\{'X', batchX\}, 2)}}}
		
		\item	\( R^2_1 = 24.5\% \) and \( R^2_2 = 13.3\% \)

		
		\item	Next: scores, loadings, SPE, \( T^2 \): all the usual PCA tools
		
	\end{itemize}

\end{frame}

\begin{frame}\frametitle{SBR: scores}
	
	\begin{itemize}
		\item	Start with 2 to 3 components: \emph{just to see what's going on}
		
		\item	{\color{myOrange}{\texttt{batch\_PCA = lvm(\{'X', batchX\}, 2)}}}
		
		\item	\( R^2_1 = 24.5\% \) and \( R^2_2 = 13.3\% \)

		
		\item	Next: scores, loadings, SPE, \( T^2 \): all the usual PCA tools
		
	\end{itemize}

\end{frame}


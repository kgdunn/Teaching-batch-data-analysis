%!TEX root = batch-course.tex

\begin{frame}\frametitle{Summary: Data understanding and troubleshooting}
	
General procedure to follow:	

\begin{itemize}
	\item	Initially include all batches
	
	\item	Build model on all data (usually 2 to 3 components)
	
	\item	Score plot to see if: \hfill {\color{myOrange}{\texttt{plot(model,'scores')}}}
	
			\begin{itemize}
				\item	Clusters due to seasonal effect?
				
				\item	Different types of abnormalities (faults) will cluster together.
				
				\item	Different product grade clusters?
				
				\item	Raw material suppler change over groupings?
				
				\item	Outliers? Usually we know which batches were bad.  We usually (always?) start to see bad batches in the scores.
				
				\item	Most informative: \( t_1 - t_2 \) and \( t_1 - t_3 \) and \( t_2 - t_3 \)
			\end{itemize}
\end{itemize}

\end{frame}

\begin{frame}\frametitle{Summary: Data understanding and troubleshooting}
	
SPE plots:

\begin{itemize}
	\item	There are 2 types:
	
			\begin{itemize}
				\item	overall (available at the end of the batch)
				
				\item	instantaneous (evolving, during the batch)
			\end{itemize}
			
	\item	Contributions to SPE show \emph{which variables} had a changed relationship
	
	\item	and at \emph{what time} during the batch
	
	\item	Should always be verified in the raw data
\end{itemize}

\end{frame}

\begin{frame}\frametitle{Summary: Data understanding and troubleshooting}
	
\begin{itemize}
	\item	Loadings (PCA) and weights (PLS):
	
			\begin{itemize}
				\item	shows \emph{which variables} and at \emph{what times} they dominate each latent variable
				
			\end{itemize}


	\item 	\( R^2 \) plots
		
			\begin{itemize}
				\item	interpreted in combination with loadings plot
				
				\item	common to see one latent variable dominating variance explained in a phase (e.g LV1 for phase 2 and LV1 in phase 3)
				
				\item	noisy tags: show low \( R^2  \) and small loadings in components at all times.  Good candidates for removal if:
				 		
						\begin{itemize}
							\item	never used in any contributions, or to explain outliers
							
							\item	unrelated to explaining quality variables
						\end{itemize}
				
				\item	large loading and high \( R^2 \): very important variable for that component
			\end{itemize}

\end{itemize}

\end{frame}

\begin{frame}\frametitle{Summary: Data understanding and troubleshooting}

Removing observations:	
\begin{itemize}
	\item	Remove observations once contributions fully understood, especially if they distort score space
	
	\item	Can be tedious, but important to learn which batch problems are prevalent
	
	\item	Once removed, only common cause batches left:
	
			\begin{itemize}
				\item	we will use that as reference database for monitoring!
			\end{itemize}

\end{itemize}

\end{frame}
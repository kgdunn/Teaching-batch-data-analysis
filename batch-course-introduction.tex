%!TEX root = batch-course.tex

%-------------------------------------------------
% Sources and resources
%-------------------------------------------------
\begin{frame}\frametitle{Sources and resources}

These notes are based on ConnectMV's experience with batch systems, but also the extensive research into batch data analysis done at McMaster University

\begin{enumerate}

	\item	Nomikos, ``Statistical Process Control of Batch Processes'', Ph.D thesis, McMaster University, 1995, {\scriptsize http://digitalcommons.mcmaster.ca/opendissertations/1667/}
	
	%\item	Garcia-Mu\~noz, ``Batch process improvement using latent latent variable methods'', Ph.D thesis, McMaster University, 2004 {\scriptsize http://digitalcommons.mcmaster.ca/opendissertations/1596/}
	
	\item	Wold, Kettaneh-Wold, MacGregor, Dunn, ``Batch Process Modeling and MSPC'', \emph{Comprehensive Chemometrics}, 2009, {\scriptsize http://dx.doi.org/10.1016/B978-044452701-1.00108-3}
		
	\item	And many other batch papers - please ask.
\end{enumerate}
\end{frame}

\begin{frame}\frametitle{}
\end{frame}

%-------------------------------------------------
\section{What do we want to learn from our data?}
%-------------------------------------------------

\begin{frame}\frametitle{What do we want to learn from our data?}

\begin{enumerate}
	\item {\bf \color{myGreen}Improve/confirm} process understanding
\begin{itemize}

	\item	confirm which phenomena have greatest effect in the data (PCA)

	\item	which variables have most influence on final quality (PLS)

	\item	interpret latent variables (all LV model types)

\end{itemize}
\end{enumerate}
\end{frame}

\begin{frame}\frametitle{What do we want to learn from our data?}

\begin{enumerate}
	\setcounter{enumi}{1}
	\item {\bf \color{myGreen}Troubleshooting}
	\begin{itemize}
		\item 	confirm a known problem occurred (SPE, \( T^2 \), \( t_a \))
		\item 	use contribution plots to diagnose
		\begin{itemize}
			\item 	what went wrong?
			\item 	when did it go wrong (batch)? 
		\end{itemize}\pause
		
		\item 	use interpretation of \( t_1, t_2 \) to explain high/low score values\pause
		\item 	use engineering judgement to fix problems from insight gained from LV outputs
	\end{itemize}
\end{enumerate}
\end{frame}
	
\begin{frame}\frametitle{What do we want to learn from our data?}

\begin{enumerate}
	\setcounter{enumi}{2}
	
	\item {\bf \color{myGreen}Improve/optimize} a process
	
	\begin{itemize}
		\item 	how can we move in the latent variable space?
		
		\item 	what causal variables can be adjusted to move in \( t_1, t_2, \ldots \)
		
		\item 	latent variable DOE's to fill in gaps in ``sparse'' models
	\end{itemize}	
	
	These all rely on the concept of \alert{``model inversion''}
	
\end{enumerate}
\end{frame}

\begin{frame}\frametitle{What do we want to learn from our data?}

\begin{enumerate}
	\setcounter{enumi}{3}
	
	\item {\bf \color{myGreen}Predictive modelling} 
	
	\begin{itemize}
		\item 	uses PLS models
		
		\item 	e.g. inferential sensors provide real-time prediction of \( \mathbf{Y} \), our final quality attributes (FQAs)
		\begin{itemize}
			\item 	PLS handles colinearity, missing values, noisy \( \mathbf{X} \)-variables
			\item 	these 3 issues cannot be dealt with by ordinary least squares
		\end{itemize}
	\end{itemize}	
\end{enumerate}
\end{frame}

\begin{frame}\frametitle{What do we want to learn from our data?}

\begin{enumerate}
	\setcounter{enumi}{4}
	\item {\bf \color{myGreen}Process monitoring} 
	
	\begin{itemize}
		
		\item 	build a model from ``in control'' operation: make sure we remain there
		
		\item 	don't need to take extra action if we remain in that space
		
		\item 	do adjust process if trending out of space; we can monitor: 
		
		\begin{itemize}
			\item 	SPE, 
			
			\item	\( T^2 \)
			
			\item	\( t_a \) score limits 
			
		\end{itemize}
		
		\item 	use contribution plots to diagnose problems
	\end{itemize}	
\end{enumerate}
\end{frame}

\begin{frame}\frametitle{What do we want to learn from our data?}

\begin{enumerate}
	\setcounter{enumi}{4}
	\item 	{\bf \color{myGreen}Process monitoring} (cont'd)	

	\vspace{10pt}	
			Advantages over \emph{univariate} monitoring:
		
			\begin{itemize}
				\item 	we monitor \textbf{many} raw variables with summarized scores, SPE, and \( T^2 \)
				\item 	i.e. we actually have \emph{fewer} control charts with multivariate monitoring
				\item 	multivariate charts interpreted in the same way as univariate charts
				\item 	i.e. no extra operator training required
				\item 	ordinary Shewhart charts do not give any help with diagnosis
			\end{itemize}	
\end{enumerate}
\end{frame}

\begin{frame}\frametitle{Tools to learn from our data}

\begin{enumerate}
	\item 	\alert{Generic}: applicable to all types of processes
	\item 	\alert{Informative}: easy to use by untrained staff for decision making
		\begin{itemize}
			\item clear and quick action possible
		\end{itemize}
	\item 	\alert{Simple to implement}: straightforward calculations and fast
\end{enumerate}

\vspace{1cm}

\begin{exampleblock}{}<2->
\centering{\large{\color{myOrange} Latent variable methods meet these criteria}}
\end{exampleblock}
\end{frame}